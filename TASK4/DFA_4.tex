\documentclass[a4paper, 12pt]{article}
\usepackage[utf8]{inputenc}

\usepackage[left=1.5cm,right=1cm,
    top=3cm,bottom=3cm, bindingoffset=0cm]{geometry}
    
\usepackage[autosize]{dot2texi}
\usepackage{tikz}
\usetikzlibrary{shapes,arrows}

\usepackage[T2A]{fontenc} 
\usepackage[utf8]{inputenc}
\usepackage[english, russian]{babel} 

\usepackage[usenames]{color}
\usepackage{colortbl}

\usepackage{amsmath, amsfonts, amssymb, amsthm, mathtools}
\title{DFA 4}
\author{Руслан Кутдусов А-13а}
\date{April 2022}

\begin{document}

\maketitle

\section*{1}
\[ L = \{ (aab)^nb(aba)^m \quad | \quad n \geq 0, m \geq 0 \} \]

\begin{center}
    \begin{dot2tex}
        digraph G {
            rankdir=LR;
            node [shape = none]; "";
            node [shape = doublecircle]; t;
            node [shape = circle];
            "" -> s;
            s -> 1 [label = a];
            1 -> 2 [label = a];
            2 -> s [label = b];
            s -> t [label = b];
            t -> 3 [label = a];
            3 -> 4 [label = b];
            4 -> t [label = a];
        }
    \end{dot2tex}
\end{center}

Язык является регулярным.

\section*{2}
\[ L = \{ uaav | u \in \{a, b\}^*, v \in \{ a, b \}^*, |u|_b \geq |v|_a \}  \]

Лемма о накачке является необходимым условием регулярности языка. Используем отрицание леммы для доказательства нерегулярности заданного языка.

Зафиксируем $ p \in \mathbb{N} $, возьмем $ w = b^paaa^p, |w| \geq p $. Разобьем $ w = xyz $ так, что $ | xy | \leq p, y \neq \lambda $. Пусть $ x = b^{\alpha}, |x| = \alpha $ и $ y = b^{\beta}, |y| = \beta $, причем $ \alpha + \beta \leq p, \beta \neq 0 $. Тогда $ z = b^{p - \alpha - \beta}aaa^p $, поэтому $ w = b^{\alpha}b^{\beta}b^{p - \alpha - \beta}aaa^p $. При $ i = 0 $ имеем $ w = b^{\alpha}b^{0}b^{p - \alpha - \beta}aaa^p = b^{p - \beta}aaa^p \notin L $.

\section*{3}
\[ L = \{ a^mw | w \in \{a, b\}^*, 1 \leq  |w|_b \leq m \}  \]

По теореме о замкнутости регулярных языков относительно различных операций, отрицание регулярного языка также является регулярным языком. Используем отрицание леммы о разрастании для доказательства нерегулярности отрицания заданного языка:

\[ \overline{L} = \{ a^mw | w \in \{a, b\}^*, m \geq 0, |w|_b > m \}  \]

Зафиксируем $ p \in \mathbb{N} $, возьмем $ w_1 = a^pb^{p+1}, |w_1| \geq p $. Разобьем $ w_1 = xyz $ так, что $ | xy | \leq p, y \neq \lambda $. Пусть $ x = a^{\alpha}, |x| = \alpha $ и $ y = a^{\beta}, |y| = \beta $, причем $ \alpha + \beta \leq p, \beta \neq 0 $. Тогда $ z = a^{p - \alpha - \beta}b^{p+1} $, поэтому $ w_1 = a^{\alpha}a^{\beta}a^{p - \alpha - \beta}b^{p + 1} $. При $ i > 1 $ (допустим, $ i = 2 $) имеем $ w_1 = a^{\alpha}(a^{\beta})^2a^{p - \alpha - \beta}b^{p + 1} = a^{p + \beta}b^{p + 1} \notin \overline{L} $ (у нас $ \beta > 0 $).

\section*{4}
\[ L = \{ a^kb^ma^n | k = n \vee m > 0 \}  \]

\begin{enumerate}
    \item $ k = n $
    
    Используем отрицание леммы о накачке для доказательства нерегулярности языка $ L_1 = \{ a^kb^ma^n | k = n \}  $.
    
    Зафиксируем $ p \in \mathbb{N} $, возьмем $ w = a^pb^ma^p, |w| \geq p $. Разобьем $ w = xyz $ так, что $ | xy | \leq p, y \neq \lambda $. Пусть $ x = a^{\alpha}, |x| = \alpha $ и $ y = a^{\beta}, |y| = \beta $, причем $ \alpha + \beta \leq p, \beta \neq 0 $. Тогда $ z = a^{p - \alpha - \beta}b^ma^p $, поэтому $ w = a^{\alpha}a^{\beta}a^{p - \alpha - \beta}b^ma^p $. При $ i \neq 1 $ (допустим, $ i = 2 $) имеем $ w = a^{\alpha}(a^{\beta})^ia^{p - \alpha - \beta}b^ma^p = a^{p + \beta}b^ma^p \notin L $ (у нас $ \beta \neq 0 $).
    
    \item $ m > 0 $
    
    Язык $ L_2 = \{ a^kb^ma^n | m > 0 \}  $ является регулярным.
    
    \begin{center}
    \begin{dot2tex}
        digraph G1 {
            rankdir=LR;
            node [shape = none]; "";
            node [shape = doublecircle]; q; t;
            node [shape = circle];
            "" -> s;
            s -> s [label = a];
            s -> q [label = b];
            q -> q [label = b];
            q -> t [label = a];
            t -> t [label = a]
        }
    \end{dot2tex}
\end{center}
    
\end{enumerate}

Можно сделать вывод о том, что язык $ L $ яляется регулярным.

\section*{5}
\[ L = \{ ucv | u \in \{a, b\}^*, v \in \{a, b\}^*∗, u \neq v^{Reverse}\}  \]

По теореме о замкнутости регулярных языков относительно различных операций, отрицание регулярного языка также является регулярным языком. Используем отрицание леммы о разрастании для доказательства нерегулярности отрицания заданного языка:

\[ \overline{L} = \{ ucv | u \in \{a, b\}^*, v \in \{a, b\}^*∗, u = v^{Reverse}\}  \]

Зафиксируем $ p \in \mathbb{N} $, возьмем $ w = a^pbcba^p, |w| \geq p $. Разобьем $ w = xyz $ так, что $ | xy | \leq p, y \neq \lambda $. Пусть $ x = a^{\alpha}, |x| = \alpha $ и $ y = a^{\beta}, |y| = \beta $, причем $ \alpha + \beta \leq p, \beta \neq 0 $. Тогда $ z = a^{p - \alpha - \beta}bcba^p $, поэтому $ w = a^{\alpha}a^{\beta}a^{p - \alpha - \beta}bcba^p $. При $ i = 0 $ имеем $ w = a^{\alpha}(a^{\beta})^0a^{p - \alpha - \beta}bcba^p = a^{p - \beta}bcba^p \notin \overline{L} $ (у нас $ \beta \neq 0 $).

\end{document}
