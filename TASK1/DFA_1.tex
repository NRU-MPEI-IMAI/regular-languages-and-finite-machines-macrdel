\documentclass[a4paper, 12pt]{article}
\usepackage[utf8]{inputenc}

\usepackage[left=1.5cm,right=1.5cm,
    top=2cm,bottom=2cm, bindingoffset=0cm]{geometry}
    
\usepackage[]{dot2texi}
\usepackage{tikz}
\usetikzlibrary{shapes,arrows}

\usepackage[T2A]{fontenc} %кодировка
\usepackage[utf8]{inputenc} %кодировка исходного текста
\usepackage[english, russian]{babel} %локализация и переносы

%Математика
\usepackage{amsmath, amsfonts, amssymb, amsthm, mathtools}

\title{DFA 1}
\author{Руслан Кутдусов А-13а}
\date{March 2022}

\begin{document}

\maketitle

\section*{Task}
Ответом на данное задание является конечный автомат, распознающий описанный язык.
Автомат должен быть детерминированным.

\section*{1}
\[ L = \{ w \in \{a, b, c\}^* \quad | \quad |w|_c = 1 \} \]

\begin{center}
\begin{dot2tex}%[pgf]
digraph DFA1
{
rankdir=LR;
node [shape = none]; "";
node [shape = doublecircle]; q2;
node [shape = circle];
"" -> q1;
q1 -> q1 [label = "a, b"];
q1 -> q2 [label = "c"];
q2 -> q2 [label = "a, b"]
}
\end{dot2tex}
\end{center}

\section*{2}

\[
\begin{aligned}
L &= \{ w \in \{a, b\}^* \quad | \quad |w|_a \leq 2, |w|_b \geq 2 \} \\ 
L_1 &= \{ w \in \{a, b\}^* \quad | \quad |w|_a \leq 2 \}
\end{aligned}
\]

\begin{center}
\begin{dot2tex}%[pgf]
digraph DFA21
{
rankdir=LR;
node [shape = none]; "";
node [shape = doublecircle];
"" -> s1;
s1 -> s1 [label = "b"];
s1 -> t11 [label = "a"];
t11 -> t11 [label = "b"];
t11 -> t12 [label = "a"];
t12 -> t12 [label = "b"]
}
\end{dot2tex}
\end{center}

\[ 
\begin{aligned}
L_2 = \{ w \in \{a, b\}^* \quad | \quad |w|_b \geq 2 \}
\end{aligned}
\]

\begin{center}
\begin{dot2tex}%[pgf]
digraph DFA22
{
rankdir=LR;
node [shape = none]; "";
node [shape = doublecircle]; t2;
node [shape = circle];
"" -> s2;
s2 -> s2 [label = "a"];
s2 -> q2 [label = "b"];
q2 -> q2 [label = "a"];
q2 -> t2 [label = "b"];
t2 -> t2 [label = "a, b"]
}
\end{dot2tex}
\end{center}

\newpage

\[
\begin{aligned}
L &= L_1 \cap L_2 \\
A_1 &= \{ \Sigma = \{a, b\}, Q_1 = \{ s_1, t_{11}, t_{12} \}, s_1, T_1 = \{ s_1, t_{11}, t_{12} \}, \delta_1  \} \\
A_2 &= \{ \Sigma = \{a, b\}, Q_2 = \{ s_2, q_2, t_2 \}, s_2, T_2 = \{ t_2 \}, \delta_2  \} \\
A &= \{ \Sigma, Q, s, T, \delta \}
\end{aligned}
\]


\begin{enumerate}
    \item $ \Sigma = \{ a, b \} $
    \item $ Q = Q_1 \times Q_2 = \{ \langle s_1, s_2 \rangle, \langle s_1, q_2 \rangle, \langle s_1, t_2 \rangle, \langle t_{11}, s_2 \rangle, \langle t_{11}, q_2 \rangle, \langle t_{11}, t_2 \rangle, \langle t_{12}, s_2 \rangle, \langle t_{12}, q_2 \rangle, \langle t_{12}, t_2 \rangle \} $
    \item $ s = \langle s_1, s_2 \rangle $
    \item $ T = T_1 \times T_2 = \{ \langle s_1, t_2 \rangle, \langle t_{11}, t_2 \rangle, \langle t_{12}, t_2 \rangle \}$
    \item $ \delta : $
    \begin{itemize}
        \item $ \delta (\langle s_1, s_2 \rangle, a) = \langle \delta_1 (s_1, a), \delta_2 (s_2, a) \rangle = \langle t_{11}, s_2 \rangle $
        \item $ \delta (\langle s_1, s_2 \rangle, b) = \langle \delta_1 (s_1, b), \delta_2 (s_2, b) \rangle = \langle s_1, q_2 \rangle $
        \item $ \delta (\langle s_1, q_2 \rangle, a) = \langle \delta_1 (s_1, a), \delta_2 (q_2, a) \rangle = \langle t_{11}, q_2 \rangle $
        \item $ \delta (\langle s_1, q_2 \rangle, b) = \langle \delta_1 (s_1, b), \delta_2 (q_2, b) \rangle = \langle s_1, t_2 \rangle $
        \item $ \delta (\langle s_1, t_2 \rangle, a) = \langle \delta_1 (s_1, a), \delta_2 (t_2, a) \rangle = \langle t_{11}, t_2 \rangle $
        \item $ \delta (\langle s_1, t_2 \rangle, b) = \langle \delta_1 (s_1, b), \delta_2 (t_2, b) \rangle = \langle s_1, t_2 \rangle $
        
        \item $ \delta (\langle t_{11}, s_2 \rangle, a) = \langle \delta_1 (t_{11}, a), \delta_2 (s_2, a) \rangle = \langle t_{12}, s_2 \rangle $
        \item $ \delta (\langle t_{11}, s_2 \rangle, b) = \langle \delta_1 (t_{11}, b), \delta_2 (s_2, b) \rangle = \langle t_{11}, q_2 \rangle $
        \item $ \delta (\langle t_{11}, q_2 \rangle, a) = \langle \delta_1 (t_{11}, a), \delta_2 (q_2, a) \rangle = \langle t_{12}, q_2 \rangle $
        \item $ \delta (\langle t_{11}, q_2 \rangle, b) = \langle \delta_1 (t_{11}, b), \delta_2 (q_2, b) \rangle = \langle t_{11}, t_2 \rangle $
        \item $ \delta (\langle t_{11}, t_2 \rangle, a) = \langle \delta_1 (t_{11}, a), \delta_2 (t_2, a) \rangle = \langle t_{12}, t_2 \rangle $
        \item $ \delta (\langle t_{11}, t_2 \rangle, b) = \langle \delta_1 (t_{11}, b), \delta_2 (t_2, b) \rangle = \langle t_{11}, t_2 \rangle $
        
        \item $ \delta (\langle t_{12}, s_2 \rangle, a) = \langle \delta_1 (t_{12}, a), \delta_2 (s_2, a) \rangle = \o $
        \item $ \delta (\langle t_{12}, s_2 \rangle, b) = \langle \delta_1 (t_{12}, b), \delta_2 (s_2, b) \rangle = \langle t_{12}, q_2 \rangle $
        \item $ \delta (\langle t_{12}, q_2 \rangle, a) = \langle \delta_1 (t_{12}, a), \delta_2 (q_2, a) \rangle = \o $
        \item $ \delta (\langle t_{12}, q_2 \rangle, b) = \langle \delta_1 (t_{12}, b), \delta_2 (q_2, b) \rangle = \langle t_{12}, t_2 \rangle $
        \item $ \delta (\langle t_{12}, t_2 \rangle, a) = \langle \delta_1 (t_{12}, a), \delta_2 (t_2, a) \rangle = \o $
        \item $ \delta (\langle t_{12}, t_2 \rangle, b) = \langle \delta_1 (t_{12}, b), \delta_2 (t_2, b) \rangle = \langle t_{12}, t_2 \rangle $
    \end{itemize}
\end{enumerate}

We have the following automaton:

\begin{flushleft}
\begin{dot2tex}%[pgf]
digraph DFA2
{
rankdir=LR;
node [shape = doublecircle]; s1t2 t11t2 t12t2;
node [shape=none]; "";
node [shape = circle];
"" -> s1s2;
s1s2 -> t11s2 [label = "a"];
s1s2 -> s1q2 [label = "b"];
s1q2 -> t11q2 [label = "a"];
s1q2 -> s1t2 [label = "b"];
s1t2 -> t11t2 [label = "a"];
s1t2 -> s1t2 [label = "b"];
t11s2 -> t12s2 [label = "a"];
t11s2 -> t11q2 [label = "b"];
t11q2 -> t12q2 [label = "a"];
t11q2 -> t11t2 [label = "b"];
t11t2 -> t12t2 [label = "a"];
t11t2 -> t11t2 [label = "b"];
t12s2 -> t12q2 [label = "b"];
t12q2 -> t12t2 [label = "b"];
t12t2 -> t12t2 [label = "b"]
}
\end{dot2tex}
\end{flushleft}

\section*{3}
\[
\begin{aligned}
L &= \{ w \in \{a, b\}^* \quad | \quad |w|_a \neq |w|_b \} \\ 
L_1 &= \{ w \in \{a, b\}^* \quad | \quad |w|_a \text{ четно}, |w|_b \text{ нечетно} \} 
\end{aligned}
\]

\[ L_{11} = \{ w \in \{a, b\}^* \quad | \quad |w|_a \text{ четно} \} \]

\begin{center}
\begin{dot2tex}
digraph DFA311 {
rankdir = LR;
node [shape = doublecircle]; s11;
node [shape=none]; "";
node [shape = circle];
"" -> s11;
s11 -> s11 [label = "b"];
s11 -> t11 [label = "a"];
t11 -> t11 [label = "b"];
t11 -> s11 [label = "a"]
}
\end{dot2tex}
\end{center}

\[
L_{12} = \{ w \in \{a, b\}^* \quad | \quad |w|_b \text{ нечетно} \}
\]

\begin{center}
\begin{dot2tex}
digraph DFA312 {
rankdir = LR;
node [shape=none]; "";
node [shape = doublecircle]; t12;
node [shape = circle];
"" -> s12;
s12 -> s12 [label = "a"];
s12 -> t12 [label = "b"];
t12 -> t12 [label = "a"];
t12 -> s12 [label = "b"]
}
\end{dot2tex}
\end{center}

\[
\begin{aligned}
L_1 &= L_{11} \cap L_{12} \\
\Sigma_1 &= \{a, b \} \\
Q_1 &= Q_{11} \times Q_{12} = \{  \langle s_{11}, s_{12} \rangle, \langle s_{11}, t_{12} \rangle, \langle t_{11}, s_{12} \rangle, \langle t_{11}, t_{12} \rangle  \} \\
s_1 &= \langle s_{11}, s_{12} \rangle \\
T_1 &= T_{11} \times T_{12} = \{ \langle s_{11}, t_{12} \rangle \} \\
\delta_1& :
\end{aligned}
\]


\begin{itemize}
    \item $ \delta_1 (\langle s_{11}, s_{12} \rangle, a) = \langle \delta_{11} (s_{11}, a), \delta_{12} (s_{12}, a) \rangle = \langle t_{11}, s_{12} \rangle $
    \item $ \delta_1 (\langle s_{11}, s_{12} \rangle, b) = \langle \delta_{11} (s_{11}, b), \delta_{12} (s_{12}, b) \rangle = \langle s_{11}, t_{12} \rangle $
    \item $ \delta_1 (\langle s_{11}, t_{12} \rangle, a) = \langle \delta_{11} (s_{11}, a), \delta_{12} (t_{12}, a) \rangle = \langle t_{11}, t_{12} \rangle $
    \item $ \delta_1 (\langle s_{11}, t_{12} \rangle, b) = \langle \delta_{11} (s_{11}, b), \delta_{12} (t_{12}, b) \rangle = \langle s_{11}, s_{12} \rangle $
    \item $ \delta_1 (\langle t_{11}, s_{12} \rangle, a) = \langle \delta_{11} (t_{11}, a), \delta_{12} (s_{12}, a) \rangle = \langle s_{11}, s_{12} \rangle $
    \item $ \delta_1 (\langle t_{11}, s_{12} \rangle, b) = \langle \delta_{11} (t_{11}, b), \delta_{12} (s_{12}, b) \rangle = \langle t_{11}, t_{12} \rangle $
    \item $ \delta_1 (\langle t_{11}, t_{12} \rangle, a) = \langle \delta_{11} (t_{11}, a), \delta_{12} (t_{12}, a) \rangle = \langle s_{11}, t_{12} \rangle $
    \item $ \delta_1 (\langle t_{11}, t_{12} \rangle, b) = \langle \delta_{11} (t_{11}, b), \delta_{12} (t_{12}, b) \rangle = \langle t_{11}, s_{12} \rangle $
\end{itemize}

Имеем:

\begin{dot2tex}
digraph DFA31 {
rankdir = LR;
node [shape=none]; "";
node [shape = doublecircle]; s11t12;
node [shape = circle];
"" -> s11s12;
s11s12 -> t11s12 [label = "a"];
s11s12 -> s11t12 [label = "b"];
t11s12 -> s11s12 [label = "a"];
t11s12 -> t11t12 [label = "b"];
t11t12 -> s11t12 [label = "a"];
t11t12 -> t11s12 [label = "b"];
s11t12 -> t11t12 [label = "a"];
s11t12 -> s11s12 [label = b]
}
\end{dot2tex}

\[ L_2 = \{ w \in \{a, b\}^* \quad | \quad |w|_b \text{ четно}, |w|_a \text{ нечетно} \} \]

Аналогично:

\begin{dot2tex}
digraph DFA32 {
rankdir = LR;
node [shape=none]; "";
node [shape = doublecircle]; s21t22;
node [shape = circle];
"" -> s21s22;
s21s22 -> t21s22 [label = "b"];
s21s22 -> s21t22 [label = "a"];
t21s22 -> s21s22 [label = "b"];
t21s22 -> t21t22 [label = "a"];
t21t22 -> s21t22 [label = "b"];
t21t22 -> t21s22 [label = "a"];
s21t22 -> t21t22 [label = "b"];
s21t22 -> s21s22 [label = a]
}
\end{dot2tex}

Имеем два автомата, которые распознают разные длины последовательностей символов. Но в общем случае построить ДКА, который распознает разные длины последовательностей символов (например, обе длины четные или обе длины нечетные) нельзя. Также как это нельзя сделать для языка, в котором требуются одинаковые длины. Такие языки являются нерегулярными. ДКА не способен считать, поскольку он должен иметь состояние для представления любого возможного количества символов "a" или символов \text{"b"} , а это означает, что ему потребуется бесконечное число состояний. То есть для постороения такого ДКА требуются дополнительные ограничения. 

\section*{4}
\[ L = \{ w \in \{ a, b \}^* \quad | \quad ww = www \} \]

Одно слово удовлетворяет данному языку $ - $ $ w = \lambda $. Имеем следующее справедливое свойство конкатенации строк: $ \lambda w = w \lambda = w $. Тогда, $ ww = \lambda ww = w \lambda w = ww \lambda = www $.

\begin{center}
\begin{dot2tex}
digraph DFA4 {
rankdir = LR;
node [shape=none]; "";
node [shape = doublecircle]; st0;
"" -> st0;
st0 -> st0
}
\end{dot2tex}
\end{center}

\end{document}