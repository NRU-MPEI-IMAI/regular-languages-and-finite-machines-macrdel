\documentclass[a4paper, 12pt]{article}
\usepackage[utf8]{inputenc}

\usepackage[left=1.5cm,right=1cm,
    top=3cm,bottom=3cm, bindingoffset=0cm]{geometry}
    
\usepackage[autosize]{dot2texi}
\usepackage{tikz}
\usetikzlibrary{shapes,arrows}

\usepackage[T2A]{fontenc} 
\usepackage[utf8]{inputenc}
\usepackage[english, russian]{babel} 

\usepackage[usenames]{color}
\usepackage{colortbl}

\usepackage{amsmath, amsfonts, amssymb, amsthm, mathtools}
\title{DFA 3}
\author{Руслан Кутдусов А-13а}
\date{April 2022}

\begin{document}

\maketitle

\section*{Task}
Ответом на данное задание является минимальный ДКА, который допускает тот же язык, что описывается регулярным выражением.

\section*{1}
\[ (ab + aba)^*a \]

Построим автомат с "пустыми" переходами ( $ \lambda $-переходами ):

\begin{center}
    \begin{dot2tex}[options={--graphstyle "scale=0.8"}]
        digraph DFA11 
        {
            rankdir=LR;
            node [shape = none]; "";
            node [shape = doublecircle]; t;
            node [shape = circle];
            "" -> s;
            s -> q1;
            q1 -> q2;
            q2 -> q3 [label=a];
            q3 -> q4 [label=b];
            q4 -> q5;
            q1 -> q6;
            q6 -> q7 [label=a];
            q7 -> q8 [label=b];
            q8 -> q9 [label=a];
            q9 -> q5;
            q5 -> q1;
            q5 -> q10;
            s -> q10;
            q10 -> t [label=a]
        }
    \end{dot2tex}
\end{center}

Объединим $ \{s, q_1, q_2, q_6, q_4, q_9, q_5, q_{10}\} \rightarrow s $. Конечными сделаем $ \{ q_3, q_7  \}  $. Получим НКА.

\begin{center}
    \begin{dot2tex}[options={--graphstyle "scale=0.8"}]
        digraph NFA12 
        {
            rankdir=LR;
            node [shape = none]; "";
            node [shape = doublecircle]; q3 q7;
            node [shape = circle];
            "" -> s;
            s -> q3 [label=a];
            s -> q7 [label=a];
            q7 -> q8 [label=b];
            q8 -> s [label=a];
            q3 -> s [label=b]
        }
    \end{dot2tex}
\end{center}

Применим алгоритм Томпсона для преобразования НКА в ДКА. Из вершины $ s : a \rightarrow q_{37} , b \rightarrow \o $. Из вершины $ q_{37} : a \rightarrow \o, b \rightarrow sq_8 $. Из вершины $ sq_8 : a \rightarrow q_{37}s, b \rightarrow \o $. Из вершины $ q_{37}s : a \rightarrow q_{37}, b \rightarrow sq_8 $. Других вершин нет. Конечными становятся $ q_{37}, q_{37}s $.

\begin{center}
    \begin{dot2tex}[options={--graphstyle "scale=0.8"}]
        digraph DFA13 
        {
            rankdir=LR;
            node [shape = none]; "";
            node [shape = doublecircle]; q37 q37s;
            node [shape = circle];
            "" -> s;
            s -> q37 [label=a];
            q37 -> sq8 [label=b];
            sq8 -> q37s [label=a];
            q37s -> q37 [label=a];
            q37s -> sq8 [label=b]
        }
    \end{dot2tex}
\end{center}

\section*{2}
\[ a(a(ab)^*b)^*(ab)^* \]

\begin{flushleft}
    \begin{dot2tex}[options={--graphstyle "scale=0.52"}]
        digraph DFA21 
        {
            rankdir=LR;
            node [shape = none]; "";
            node [shape = doublecircle]; t;
            node [shape = circle];
            "" -> s;
            s -> q1 [label=a];
            q1 -> q2;
            q2 -> q3 [label=a];
            q3 -> q4;
            q4 -> q5 [label=a];
            q5 -> q6 [label=b];
            q6 -> q4;
            q6 -> q7;
            q3 -> q7;
            q7 -> q8 [label=b];
            q8 -> q9;
            q8 -> q2;
            q1 -> q9;
            q9 -> q10;
            q10 -> q11 [label=a];
            q11 -> q12 [label=b];
            q12 -> q10;
            q12 -> t;
            q9 -> t
        }
    \end{dot2tex}
\end{flushleft}

Объединим $ \{q_1, q_2, q_8, q_9, q_{10}, q_{12} \} \rightarrow q_1 $. Объединим $ \{q_3, q_7, q_4, q_6 \} \rightarrow q_3 $. Конечную сделаем $ \{ q_1 \} $. Получим НКА.

\begin{center}
    \begin{dot2tex}[options={--graphstyle "scale=0.8"}]
        digraph DFA22 
        {
            rankdir=LR;
            node [shape = none]; "";
            node [shape = doublecircle]; q1;
            node [shape = circle];
            "" -> s;
            s -> q1 [label=a];
            q1 -> q3 [label=a];
            q3 -> q5 [label=a];
            q5 -> q3 [label=b];
            q3 -> q1 [label=b];
            q1 -> q11 [label=a];
            q11 -> q1 [label=b]
        }
    \end{dot2tex}
\end{center}

Объединим $ \{q_3, q_{11} \} \rightarrow q_{311} $. Получим минимальный ДКА.

\begin{center}
    \begin{dot2tex}[options={--graphstyle "scale=0.8"}]
        digraph DFA22 
        {
            rankdir=LR;
            node [shape = none]; "";
            node [shape = doublecircle]; q1;
            node [shape = circle];
            "" -> s;
            s -> q1 [label=a];
            q1 -> q311 [label=a];
            q311 -> q5 [label=a];
            q5 -> q311 [label=b];
            q311 -> q1 [label=b]
        }
    \end{dot2tex}
\end{center}

\section*{3}

\[ (a + (a + b)(a + b)b)^* \]

\begin{flushleft}
    \begin{dot2tex}[options={--graphstyle "scale=0.6"}]
        digraph DFA31 
        {
            rankdir=LR;
            node [shape = none]; "";
            node [shape = doublecircle]; t;
            node [shape = circle];
            "" -> s;
            s -> q1;
            q1 -> q2;
            q2 -> q3;
            q3 -> q4 [label=a];
            q4 -> q5;
            q2 -> q6;
            q6 -> q7 [label=b];
            q7 -> q5;
            q5 -> q8;
            q8 -> q9 [label=a];
            q9 -> q10;
            q5 -> q11;
            q11 -> q12 [label=b];
            q12 -> q10;
            q10 -> q13 [label=b];
            q13 -> q14;
            q1 -> q15;
            q15 -> q16 [label=a];
            q16 -> q14;
            q14 -> t;
            q14 -> q1;
            s -> t
        }
    \end{dot2tex}
\end{flushleft}

\[ (a + (a + b)(a + b)b)^* = (a + aab + abb + bab + bbb)^* \]

\begin{center} 
    \begin{dot2tex}[options={--graphstyle "scale=0.8"}]
        digraph DFA32 
        {
            rankdir=LR;
            node [shape = none]; "";
            node [shape = doublecircle]; s;
            node [shape = circle];
            "" -> s;
            s -> q1 [label=a];
            s -> q2 [label=b];
            q1 -> q3 [label="a,b"];
            q2 -> q3 [label="a,b"];
            q3 -> s [label=b];
            s -> s [label=a]
        }
    \end{dot2tex}
\end{center}

По алгоритму Томпсона преобразуем в ДКА. Из вершины $ s : a \rightarrow sq_{1} , b \rightarrow q_2 $. Из вершины $ sq_1 : a \rightarrow sq_{1}q_{3} , b \rightarrow q_2q_3 $. Из вершины $ q_2 : a \rightarrow q_{3} , b \rightarrow q_3 $. Из вершины $ sq_1q_3 : a \rightarrow sq_1q_{3} , b \rightarrow sq_2q_3 $. Из вершины $ q_2q_3 : a \rightarrow q_{3} , b \rightarrow sq_3 $. Из вершины $ q_3 : a \rightarrow \o , b \rightarrow s $. Из вершины $ sq_2q_3 : a \rightarrow sq_1q_{3} , b \rightarrow sq_2q_3 $. Из вершины $ sq_3 : a \rightarrow sq_1 , b \rightarrow sq_2 $. Из вершины $ sq_2 : a \rightarrow sq_1q_{3} , b \rightarrow q_2q_3 $. Конечными становятся вершины, объединенные с $ s $.

\begin{center} 
    \begin{dot2tex}[options={--graphstyle "scale=0.8"}]
        digraph DFA33 
        {
            rankdir=LR;
            node [shape = none]; "";
            node [shape = doublecircle]; s sq1 sq3 sq2 sq1q3 sq2q3;
            node [shape = circle];
            "" -> s;
            s -> sq1 [label=a];
            s -> q2 [label=b];
            sq1 -> sq1q3 [label=a];
            sq1 -> q2q3 [label=b];
            q2 -> q3 [label="a,b"];
            sq1q3 -> sq1q3 [label=a];
            sq1q3 -> sq2q3 [label=b];
            q2q3 -> q3 [label=a];
            q2q3 -> sq3 [label=b];
            q3 -> s [label=b];
            sq2q3 -> sq1q3 [label=a];
            sq2q3 -> sq2q3 [label=b];
            sq3 -> sq1 [label=a];
            sq3 -> sq2 [label=b];
            sq2 -> sq1q3 [label=a];
            sq2 -> q2q3 [label=b]
        }
    \end{dot2tex}
\end{center}

Объединим $ \{ sq_2q_3, sq_2q_3 \} \rightarrow sq_2q_3 $. Объединим $ \{ sq_1, sq_2 \} \rightarrow sq_1 $. Получим минимальный ДКА.

\begin{center} 
    \begin{dot2tex}[options={--graphstyle "scale=0.8"}]
        digraph DFA34 
        {
            rankdir=LR;
            node [shape = none]; "";
            node [shape = doublecircle]; s sq1 sq3 sq1q3;
            node [shape = circle];
            "" -> s;
            s -> sq1 [label=a];
            s -> q2 [label=b];
            sq1 -> sq1q3 [label=a];
            sq1 -> q2q3 [label=b];
            q2 -> q3 [label="a,b"];
            sq1q3 -> sq1q3 [label="a,b"];
            q2q3 -> q3 [label=a];
            q2q3 -> sq3 [label=b];
            q3 -> s [label=b];
            sq3 -> sq1 [label="a,b"];
        }
    \end{dot2tex}
\end{center}

\section*{4}

\[ (b + c)((ab)^*c + (ba)^*)^* \]

\begin{flushleft} 
    \begin{dot2tex}[options={--graphstyle "scale=0.6"}]
        digraph DFA41 
        {
            rankdir=LR;
            node [shape = none]; "";
            node [shape = doublecircle]; t;
            node [shape = circle];
            "" -> s;
            s -> 1;
            1 -> 2 [label=b];
            2 -> 3;
            s -> 4;
            4 -> 5 [label=c];
            5 -> 3;
            3 -> 6;
            6 -> 7;
            7 -> 8;
            8 -> 9 [label=a];
            9 -> 10 [label=b];
            10 -> 8;
            10 -> 11;
            7 -> 11;
            11 -> 12 [label=c];
            12 -> 13;
            13 -> 6;
            13 -> t;
            6 -> 14;
            14 -> 15;
            15 -> 16 [label=b];
            16 -> 17 [label=c];
            17 -> 15;
            17 -> 18;
            14 -> 18;
            18 -> 13;
            3 -> t
        }
    \end{dot2tex}
\end{flushleft}

Построим ДКА.

\begin{center}
    \begin{dot2tex} [options={--graphstyle "scale=0.8"}]
        digraph DFA42 
        {
            rankdir=LR;
            node [shape = none]; "";
            node [shape = doublecircle]; 1 2;
            node [shape = circle];
            "" -> s;
            s -> 1 [label = "b,c"];
            1 -> 2 [label = c];
            2 -> 2 [label = c];
            1 -> 3 [label = a];
            3 -> 4 [label = b];
            4 -> 3 [label = a];
            4 -> 2 [label = c];
            1 -> 5 [label = b];
            5 -> 1 [label = a]
        }
    \end{dot2tex}
\end{center}

Минимизируем ДКА (объединим $ \{ 1, 2 \} $).

\begin{center}
    \begin{dot2tex} [options={--graphstyle "scale=0.8"}]
        digraph DFA43 
        {
            rankdir=LR;
            node [shape = none]; "";
            node [shape = doublecircle]; 12;
            node [shape = circle];
            "" -> s;
            s -> 12 [label = "b,c"];
            12 -> 12 [label = c];
            12 -> 3 [label = a];
            3 -> 4 [label = b];
            4 -> 3 [label = a];
            4 -> 12 [label = c];
            12 -> 5 [label = b];
            5 -> 12 [label = a]
        }
    \end{dot2tex}
\end{center}

\section*{5}

\[ (a + b)^+(aa + bb + abab + baba)(a + b)^+ \]

Построим сразу НКА.

\begin{center}
    \begin{dot2tex} [options={--graphstyle "scale=0.8"}]
        digraph DFA51 
        {
            rankdir=LR;
            node [shape = none]; "";
            node [shape = doublecircle]; t;
            node [shape = circle];
            "" -> s;
            s -> 1 [label = "a,b"];
            1 -> 1 [label = "a,b"];
            1 -> 2 [label = a];
            2 -> 3 [label = b];
            3 -> 4 [label = a];
            1 -> 5 [label = b];
            5 -> 6 [label = a];
            6 -> 7 [label = b];
            2 -> 8 [label = a];
            4 -> 8 [label = b];
            5 -> 8 [label = b];
            7 -> 8 [label = a];
            8 -> t [label = "a,b"];
            t -> t [label = "a,b"]
        }
    \end{dot2tex}
\end{center}

\begin{center}
\begin{tabular}{||c | c | c ||}
    \hline
  \rowcolor[gray]{.9} node & a & b \\  \hline \hline
  s & 1 & 1 \\ \hline
  1 & 12 & 15 \\ \hline
  12 & 128 & 153 \\ \hline
  15 & 126 & 158  \\ \hline
  128 & 128t & 153t \\ \hline
  153 & 1264 & 158 \\ \hline
  126 & 128 & 1537 \\ \hline
  158 & 126t & 158t \\ \hline
  128 & 128t & 153t \\ \hline
  153t & 1264t & 158t  \\ \hline
  1264 & 128 & 15378 \\ \hline
  1537 & 12648 & 158 \\ \hline
  126t & 128t & 1537t \\ \hline
  158t & 126t & 158t \\ \hline
  1264t & 128t & 1537t \\ \hline
  15378 & 12648t & 158t  \\ \hline
  12648 & 128t & 15378t \\ \hline
  1537t & 12648t & 158t \\ \hline
  15378t & 12648t & 158t \\ \hline
  12648t & 128t &  15378t \\ \hline
\end{tabular}
\end{center}

% \begin{flushleft}
%     \begin{dot2tex} [options={--graphstyle "scale=0.4"}]
%         digraph DFA52 
%         {
%             rankdir=LR;
% 	        node [shape = none]; "";
% 	        node [shape = doublecircle]; t128 t153 t1264 t158 t126 t1537 t15378 t12648;
% 	        node [shape = circle];
%             "" -> s;
%             s -> 1 [label = "a,b"];
%             1 -> 12 [label = a];
%             1 -> 15 [label = b];
%             12 -> 128 [label = a];
%             12 -> 153 [label = b];
%             15 -> 126 [label = a];
%             15 -> 158 [label = b];
%             128 -> t128 [label = a];
%             128 -> t153 [label = b];
%             153 -> 1264 [label = a];
%             153 -> 158 [label = b];
%             126 -> 128 [label = a];
%             126 -> 1537 [label = b];
%             158 -> t126 [label = a];
%             158 -> t158 [label = b];
%             t128 -> t128 [label = a];
%             t128 -> t153 [label = b];
%             t153 -> t1264 [label = a];
%             t153 -> t158 [label = b];
%             1264 -> 128 [label = a];
%             1264 -> 15378 [label = b];
%             1537 -> 12648 [label = a];
%             1537 -> 158 [label = b];
%             t126 -> t128 [label = a];
%             t126 -> t1537 [label = b];
%             t158 -> t126 [label = a];
%             t158 -> t158 [label = b];
%             t1264 -> t128 [label = a];
%             t1264 -> t15378 [label = b];
%             15378 -> t12648 [label = a];
%             15378 -> t158 [label = b];
%             12648 -> t128 [label = a];
%             12648 -> t15378 [label = b];
%             t1537 -> t12648 [label = a];
%             t1537 -> t158 [label = b];
%             t15378 -> t12648 [label = a];
%             t15378 -> t158 [label = b];
%             t12648 -> t128 [label = a];
%             t12648 -> t15378 [label = b]
%         }
%     \end{dot2tex}
% \end{flushleft}

\begin{figure}[h]
\center{\includegraphics[scale=0.9]{DFA352.png}}
\end{figure}

Можно объединить все конечные вершины в одну.

\begin{center}
\begin{dot2tex} [options={--graphstyle "scale=0.8"}]
        digraph DFA52 
         {
            rankdir=LR;
 	        node [shape = none]; "";
 	        node [shape = doublecircle]; t;
 	        node [shape = circle];
             "" -> s;
             s -> 1 [label = "a,b"];
             1 -> 12 [label = a];
             1 -> 15 [label = b];
             12 -> 128 [label = a];
             12 -> 153 [label = b];
             15 -> 126 [label = a];
             15 -> 158 [label = b];
             128 -> t [label = "a,b"];
             153 -> 1264 [label = a];
             153 -> 158 [label = b];
             126 -> 128 [label = a];
             126 -> 1537 [label = b];
             158 -> t [label = "a,b"];
             1264 -> 128 [label = a];
             1264 -> 15378 [label = b];
             1537 -> 12648 [label = a];
             1537 -> 158 [label = b];
             15378 -> t [label = "a,b"];
             12648 -> t [label = "a,b"];
             t -> t [label = "a,b"]
         }
\end{dot2tex}
\end{center}

На основе эквивалентности объединим вершины $ \{ 128, 15378, 158, 12648  \} \rightarrow q_{8}, \{ 1264, 1537 \} \rightarrow q_{47} \} $. Минимизированный ДКА.

\begin{center}
\begin{dot2tex} [options={--graphstyle "scale=0.8"}]
        digraph DFA52 
         {
            rankdir=LR;
 	        node [shape = none]; "";
 	        node [shape = doublecircle]; t;
 	        node [shape = circle];
            "" -> s;
             s -> 1 [label = "a,b"];
             1 -> 12 [label = a];
             1 -> 15 [label = b];
             12 -> q8 [label = a];
             12 -> 153 [label = b];
             15 -> 126 [label = a];
             15 -> q8 [label = b];
             153 -> q47 [label = a];
             153 -> q8 [label = b];
             126 -> q8 [label = a];
             126 -> q47 [label = b];
             q47 -> q8 [label = "a,b"];
             q8 -> t [label = "a,b"]
         }
\end{dot2tex}
\end{center}

\end{document}